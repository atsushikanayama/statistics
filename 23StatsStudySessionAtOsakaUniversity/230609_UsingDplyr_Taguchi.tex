% Options for packages loaded elsewhere
\PassOptionsToPackage{unicode}{hyperref}
\PassOptionsToPackage{hyphens}{url}
%
\documentclass[
]{article}
\usepackage{amsmath,amssymb}
\usepackage{iftex}
\ifPDFTeX
  \usepackage[T1]{fontenc}
  \usepackage[utf8]{inputenc}
  \usepackage{textcomp} % provide euro and other symbols
\else % if luatex or xetex
  \usepackage{unicode-math} % this also loads fontspec
  \defaultfontfeatures{Scale=MatchLowercase}
  \defaultfontfeatures[\rmfamily]{Ligatures=TeX,Scale=1}
\fi
\usepackage{lmodern}
\ifPDFTeX\else
  % xetex/luatex font selection
\fi
% Use upquote if available, for straight quotes in verbatim environments
\IfFileExists{upquote.sty}{\usepackage{upquote}}{}
\IfFileExists{microtype.sty}{% use microtype if available
  \usepackage[]{microtype}
  \UseMicrotypeSet[protrusion]{basicmath} % disable protrusion for tt fonts
}{}
\makeatletter
\@ifundefined{KOMAClassName}{% if non-KOMA class
  \IfFileExists{parskip.sty}{%
    \usepackage{parskip}
  }{% else
    \setlength{\parindent}{0pt}
    \setlength{\parskip}{6pt plus 2pt minus 1pt}}
}{% if KOMA class
  \KOMAoptions{parskip=half}}
\makeatother
\usepackage{xcolor}
\usepackage[margin=1in]{geometry}
\usepackage{color}
\usepackage{fancyvrb}
\newcommand{\VerbBar}{|}
\newcommand{\VERB}{\Verb[commandchars=\\\{\}]}
\DefineVerbatimEnvironment{Highlighting}{Verbatim}{commandchars=\\\{\}}
% Add ',fontsize=\small' for more characters per line
\usepackage{framed}
\definecolor{shadecolor}{RGB}{248,248,248}
\newenvironment{Shaded}{\begin{snugshade}}{\end{snugshade}}
\newcommand{\AlertTok}[1]{\textcolor[rgb]{0.94,0.16,0.16}{#1}}
\newcommand{\AnnotationTok}[1]{\textcolor[rgb]{0.56,0.35,0.01}{\textbf{\textit{#1}}}}
\newcommand{\AttributeTok}[1]{\textcolor[rgb]{0.13,0.29,0.53}{#1}}
\newcommand{\BaseNTok}[1]{\textcolor[rgb]{0.00,0.00,0.81}{#1}}
\newcommand{\BuiltInTok}[1]{#1}
\newcommand{\CharTok}[1]{\textcolor[rgb]{0.31,0.60,0.02}{#1}}
\newcommand{\CommentTok}[1]{\textcolor[rgb]{0.56,0.35,0.01}{\textit{#1}}}
\newcommand{\CommentVarTok}[1]{\textcolor[rgb]{0.56,0.35,0.01}{\textbf{\textit{#1}}}}
\newcommand{\ConstantTok}[1]{\textcolor[rgb]{0.56,0.35,0.01}{#1}}
\newcommand{\ControlFlowTok}[1]{\textcolor[rgb]{0.13,0.29,0.53}{\textbf{#1}}}
\newcommand{\DataTypeTok}[1]{\textcolor[rgb]{0.13,0.29,0.53}{#1}}
\newcommand{\DecValTok}[1]{\textcolor[rgb]{0.00,0.00,0.81}{#1}}
\newcommand{\DocumentationTok}[1]{\textcolor[rgb]{0.56,0.35,0.01}{\textbf{\textit{#1}}}}
\newcommand{\ErrorTok}[1]{\textcolor[rgb]{0.64,0.00,0.00}{\textbf{#1}}}
\newcommand{\ExtensionTok}[1]{#1}
\newcommand{\FloatTok}[1]{\textcolor[rgb]{0.00,0.00,0.81}{#1}}
\newcommand{\FunctionTok}[1]{\textcolor[rgb]{0.13,0.29,0.53}{\textbf{#1}}}
\newcommand{\ImportTok}[1]{#1}
\newcommand{\InformationTok}[1]{\textcolor[rgb]{0.56,0.35,0.01}{\textbf{\textit{#1}}}}
\newcommand{\KeywordTok}[1]{\textcolor[rgb]{0.13,0.29,0.53}{\textbf{#1}}}
\newcommand{\NormalTok}[1]{#1}
\newcommand{\OperatorTok}[1]{\textcolor[rgb]{0.81,0.36,0.00}{\textbf{#1}}}
\newcommand{\OtherTok}[1]{\textcolor[rgb]{0.56,0.35,0.01}{#1}}
\newcommand{\PreprocessorTok}[1]{\textcolor[rgb]{0.56,0.35,0.01}{\textit{#1}}}
\newcommand{\RegionMarkerTok}[1]{#1}
\newcommand{\SpecialCharTok}[1]{\textcolor[rgb]{0.81,0.36,0.00}{\textbf{#1}}}
\newcommand{\SpecialStringTok}[1]{\textcolor[rgb]{0.31,0.60,0.02}{#1}}
\newcommand{\StringTok}[1]{\textcolor[rgb]{0.31,0.60,0.02}{#1}}
\newcommand{\VariableTok}[1]{\textcolor[rgb]{0.00,0.00,0.00}{#1}}
\newcommand{\VerbatimStringTok}[1]{\textcolor[rgb]{0.31,0.60,0.02}{#1}}
\newcommand{\WarningTok}[1]{\textcolor[rgb]{0.56,0.35,0.01}{\textbf{\textit{#1}}}}
\usepackage{graphicx}
\makeatletter
\def\maxwidth{\ifdim\Gin@nat@width>\linewidth\linewidth\else\Gin@nat@width\fi}
\def\maxheight{\ifdim\Gin@nat@height>\textheight\textheight\else\Gin@nat@height\fi}
\makeatother
% Scale images if necessary, so that they will not overflow the page
% margins by default, and it is still possible to overwrite the defaults
% using explicit options in \includegraphics[width, height, ...]{}
\setkeys{Gin}{width=\maxwidth,height=\maxheight,keepaspectratio}
% Set default figure placement to htbp
\makeatletter
\def\fps@figure{htbp}
\makeatother
\setlength{\emergencystretch}{3em} % prevent overfull lines
\providecommand{\tightlist}{%
  \setlength{\itemsep}{0pt}\setlength{\parskip}{0pt}}
\setcounter{secnumdepth}{-\maxdimen} % remove section numbering
\ifLuaTeX
  \usepackage{selnolig}  % disable illegal ligatures
\fi
\IfFileExists{bookmark.sty}{\usepackage{bookmark}}{\usepackage{hyperref}}
\IfFileExists{xurl.sty}{\usepackage{xurl}}{} % add URL line breaks if available
\urlstyle{same}
\hypersetup{
  pdftitle={dplyr を使った分析前の準備 by 田口恵也},
  pdfauthor={まとめ by 金山篤志},
  hidelinks,
  pdfcreator={LaTeX via pandoc}}

\title{dplyr を使った分析前の準備 by 田口恵也}
\author{まとめ by 金山篤志}
\date{2023-06-09}

\begin{document}
\maketitle

{
\setcounter{tocdepth}{2}
\tableofcontents
}
\hypertarget{ux30d1ux30c3ux30b1ux30fcux30b8ux306eux30a4ux30f3ux30b9ux30c8ux30fcux30ebux3068ux8aadux307fux8fbcux307f}{%
\section{パッケージのインストールと読み込み}\label{ux30d1ux30c3ux30b1ux30fcux30b8ux306eux30a4ux30f3ux30b9ux30c8ux30fcux30ebux3068ux8aadux307fux8fbcux307f}}

最初に、分析に必要なパッケージをインストールし、読み込みます。dplyrパッケージとggplot2パッケージを使用するため、以下のコードを実行します。

\begin{Shaded}
\begin{Highlighting}[]
\CommentTok{\# dplyrパッケージとggplot2パッケージをインストール}
\FunctionTok{install.packages}\NormalTok{(}\StringTok{"dplyr"}\NormalTok{)}
\end{Highlighting}
\end{Shaded}

\begin{verbatim}
## Error in contrib.url(repos, "source"): trying to use CRAN without setting a mirror
\end{verbatim}

\begin{Shaded}
\begin{Highlighting}[]
\FunctionTok{install.packages}\NormalTok{(}\StringTok{"ggplot2"}\NormalTok{)}
\end{Highlighting}
\end{Shaded}

\begin{verbatim}
## Error in contrib.url(repos, "source"): trying to use CRAN without setting a mirror
\end{verbatim}

\begin{Shaded}
\begin{Highlighting}[]
\CommentTok{\# インストールしたパッケージを読み込む}
\FunctionTok{library}\NormalTok{(}\StringTok{"dplyr"}\NormalTok{)}
\end{Highlighting}
\end{Shaded}

\begin{verbatim}
## 
## Attaching package: 'dplyr'
\end{verbatim}

\begin{verbatim}
## The following objects are masked from 'package:stats':
## 
##     filter, lag
\end{verbatim}

\begin{verbatim}
## The following objects are masked from 'package:base':
## 
##     intersect, setdiff, setequal, union
\end{verbatim}

\begin{Shaded}
\begin{Highlighting}[]
\FunctionTok{library}\NormalTok{(}\StringTok{"ggplot2"}\NormalTok{)}
\end{Highlighting}
\end{Shaded}

\hypertarget{ux30c7ux30fcux30bfux306eux6e96ux5099}{%
\subsection{データの準備}\label{ux30c7ux30fcux30bfux306eux6e96ux5099}}

次に、分析のためのデータを準備します。今回は mpg
データセットを使用します。mpgはRに組み込まれているデータセットで、自動車の燃費に関する情報が入っています。

\begin{Shaded}
\begin{Highlighting}[]
\CommentTok{\# mpgデータセットを読み込みます}
\FunctionTok{data}\NormalTok{(mpg)}
\end{Highlighting}
\end{Shaded}

\hypertarget{ux30c7ux30fcux30bfux306eux78baux8a8d}{%
\subsection{データの確認}\label{ux30c7ux30fcux30bfux306eux78baux8a8d}}

データを読み込んだ後、まずはデータの概要を確認します。head関数を使うと、データの最初の6行を表示することができます。

\begin{Shaded}
\begin{Highlighting}[]
\CommentTok{\# mpgデータセットの最初の6行を表示します}
\FunctionTok{head}\NormalTok{(mpg)}
\end{Highlighting}
\end{Shaded}

\begin{verbatim}
## # A tibble: 6 x 11
##   manufacturer model displ  year   cyl trans      drv     cty   hwy fl    class 
##   <chr>        <chr> <dbl> <int> <int> <chr>      <chr> <int> <int> <chr> <chr> 
## 1 audi         a4      1.8  1999     4 auto(l5)   f        18    29 p     compa~
## 2 audi         a4      1.8  1999     4 manual(m5) f        21    29 p     compa~
## 3 audi         a4      2    2008     4 manual(m6) f        20    31 p     compa~
## 4 audi         a4      2    2008     4 auto(av)   f        21    30 p     compa~
## 5 audi         a4      2.8  1999     6 auto(l5)   f        16    26 p     compa~
## 6 audi         a4      2.8  1999     6 manual(m5) f        18    26 p     compa~
\end{verbatim}

mpg
データは、自動車の燃費に関する情報を持つデータセットです。データセットの各列には以下のような情報が含まれています。

\begin{itemize}
\tightlist
\item
  manufacturer: 製造メーカーの名前
\item
  model: モデル名
\item
  displ: 排気量(単位: L)
\item
  year: 製造年
\item
  cyl: シリンダー数
\end{itemize}

\hypertarget{ux30c7ux30fcux30bfux306eux52a0ux5de5}{%
\subsection{データの加工}\label{ux30c7ux30fcux30bfux306eux52a0ux5de5}}

以下のRコードは、自動車の燃費データを加工するものです。ここで使っているdplyrというパッケージは、データフレームの中身を見たり、選んだり、絞り込んだり、新しく加えたりと、データを効率的に扱うための道具がたくさん詰まっています。

このコードでは、dplyrと一緒に使われる「\%\textgreater\%」という特殊な記号が出てきます。これを「パイプ」と呼び、複数の作業をつなげて一つの流れで行うことができます。

\begin{Shaded}
\begin{Highlighting}[]
\CommentTok{\# mpgという燃費データを取り出して、それに一連の作業を施します。}
\CommentTok{\# 作業の結果はmpg\_filteredという名前で保存します。}
\NormalTok{mpg\_filtered }\OtherTok{\textless{}{-}}\NormalTok{ mpg }\SpecialCharTok{\%\textgreater{}\%}  

  \CommentTok{\# まず、必要な列だけをピックアップします。メーカー、モデル、排気量、年式、シリンダー数の情報だけを取り出します。}
  \FunctionTok{select}\NormalTok{(manufacturer, model, displ, year, cyl) }\SpecialCharTok{\%\textgreater{}\%}  

  \CommentTok{\# 次に、メーカーがアウディのものだけを選びます。}
  \FunctionTok{filter}\NormalTok{(manufacturer }\SpecialCharTok{==} \StringTok{"audi"}\NormalTok{) }\SpecialCharTok{\%\textgreater{}\%}  

  \CommentTok{\# 最後に、新しい列を追加します。新しい列の名前はcenturyで、年式を100で割った数を切り上げたものを格納します。}
  \CommentTok{\# これで2000年式の車ならcenturyは20になります。}
  \FunctionTok{mutate}\NormalTok{(}\AttributeTok{century =} \FunctionTok{ceiling}\NormalTok{(year }\SpecialCharTok{/} \DecValTok{100}\NormalTok{)) }
\end{Highlighting}
\end{Shaded}

\begin{Shaded}
\begin{Highlighting}[]
\CommentTok{\# 解説なし}
\NormalTok{mpg\_filtered }\OtherTok{\textless{}{-}}\NormalTok{ mpg }\SpecialCharTok{\%\textgreater{}\%}  
  \FunctionTok{select}\NormalTok{(manufacturer, model, displ, year, cyl) }\SpecialCharTok{\%\textgreater{}\%}  
  \FunctionTok{filter}\NormalTok{(manufacturer }\SpecialCharTok{==} \StringTok{"audi"}\NormalTok{) }\SpecialCharTok{\%\textgreater{}\%}  
  \FunctionTok{mutate}\NormalTok{(}\AttributeTok{century =} \FunctionTok{ceiling}\NormalTok{(year }\SpecialCharTok{/} \DecValTok{100}\NormalTok{)) }
\end{Highlighting}
\end{Shaded}

\begin{Shaded}
\begin{Highlighting}[]
\FunctionTok{head}\NormalTok{(mpg\_filtered)  }\CommentTok{\# 加工後のデータを表示}
\end{Highlighting}
\end{Shaded}

\begin{verbatim}
## # A tibble: 6 x 6
##   manufacturer model displ  year   cyl century
##   <chr>        <chr> <dbl> <int> <int>   <dbl>
## 1 audi         a4      1.8  1999     4      20
## 2 audi         a4      1.8  1999     4      20
## 3 audi         a4      2    2008     4      21
## 4 audi         a4      2    2008     4      21
## 5 audi         a4      2.8  1999     6      20
## 6 audi         a4      2.8  1999     6      20
\end{verbatim}

最終的に、アウディの車について、選んだ5つの情報と新しく作った「century」列が含まれたデータフレームが得られます。このように、「\%\textgreater\%」を使うことで一つ一つの作業を明確にし、繋げて一つの流れとすることができます。

\hypertarget{ux30a8ux30e9ux30fcux306eux78baux8a8d}{%
\subsection{エラーの確認}\label{ux30a8ux30e9ux30fcux306eux78baux8a8d}}

R
Markdownでは、エラーが出ても読み込むことができ、再生マークをクリックしてRコードの出力結果を表示することができます。以下のコードでエラーのテストを行います。

\begin{Shaded}
\begin{Highlighting}[]
\CommentTok{\# エラーを発生させるコード例}
\FunctionTok{unknown\_function}\NormalTok{()}
\end{Highlighting}
\end{Shaded}

\begin{verbatim}
## Error in unknown_function(): could not find function "unknown_function"
\end{verbatim}

上記のコードでは、unknown\_function()
という存在しない関数を呼び出してエラーを発生させています。

\hypertarget{ux5fdcux7528ux554fux984c}{%
\subsection{応用問題}\label{ux5fdcux7528ux554fux984c}}

次のRのコードは、自動車の燃費データ(mpg)を加工して、一部の情報だけを選び、整理するものです。

\begin{Shaded}
\begin{Highlighting}[]
\CommentTok{\# mpgという自動車の燃費データから、作業の結果をmpg\_solutionという名前で保存します。}
\NormalTok{mpg\_solution }\OtherTok{\textless{}{-}}\NormalTok{ mpg }\SpecialCharTok{\%\textgreater{}\%}  

  \CommentTok{\# まずは、必要な情報だけを選び出します。ここでは、メーカー、モデル、製造年、車の種類だけを取り出します。}
  \FunctionTok{select}\NormalTok{(manufacturer, model, year, class) }\SpecialCharTok{\%\textgreater{}\%}  

  \CommentTok{\# 次に、製造年が新しいものから順にデータを並び替えます。}
  \FunctionTok{arrange}\NormalTok{(}\FunctionTok{desc}\NormalTok{(year)) }\SpecialCharTok{\%\textgreater{}\%}  

  \CommentTok{\# 最後に、取り出した情報の順番を、車の種類、メーカー、製造年、モデルの順に並べ替えます。}
  \FunctionTok{select}\NormalTok{(class, manufacturer, year, model)  }
\end{Highlighting}
\end{Shaded}

\begin{Shaded}
\begin{Highlighting}[]
\CommentTok{\# 解説なし}
\NormalTok{mpg\_solution }\OtherTok{\textless{}{-}}\NormalTok{ mpg }\SpecialCharTok{\%\textgreater{}\%}  
  \FunctionTok{select}\NormalTok{(manufacturer, model, year, class) }\SpecialCharTok{\%\textgreater{}\%}  
  \FunctionTok{arrange}\NormalTok{(}\FunctionTok{desc}\NormalTok{(year)) }\SpecialCharTok{\%\textgreater{}\%} 
  \FunctionTok{select}\NormalTok{(class, manufacturer, year, model)  }
\end{Highlighting}
\end{Shaded}

\begin{Shaded}
\begin{Highlighting}[]
\FunctionTok{head}\NormalTok{(mpg\_solution)  }\CommentTok{\# 解答を表示}
\end{Highlighting}
\end{Shaded}

\begin{verbatim}
## # A tibble: 6 x 4
##   class   manufacturer  year model     
##   <chr>   <chr>        <int> <chr>     
## 1 compact audi          2008 a4        
## 2 compact audi          2008 a4        
## 3 compact audi          2008 a4        
## 4 compact audi          2008 a4 quattro
## 5 compact audi          2008 a4 quattro
## 6 compact audi          2008 a4 quattro
\end{verbatim}

このコードを使うと、データの中から自分が必要とする情報だけを選び出し、その情報を整理して新しいデータを作ることができます。そして、その作業を一つずつ明確に行うことで、何をしているのかがわかりやすくなります。

\hypertarget{dplyrux306eux4fbfux5229ux306aux95a2ux6570group_byux3068full_join}{%
\subsection{dplyrの便利な関数:group\_by()とfull\_join()}\label{dplyrux306eux4fbfux5229ux306aux95a2ux6570group_byux3068full_join}}

\hypertarget{group_by}{%
\subsubsection{group\_by()}\label{group_by}}

ここではgroup\_by()という関数を使ったデータのグループ化方法とその結果を計算する方法を説明します。

まず、例として都市名とその人口を含む簡単なデータを考えます。

\begin{Shaded}
\begin{Highlighting}[]
\CommentTok{\# "東京"と"大阪"という二つの都市の人口データを作成します。}
\CommentTok{\# "東京"の人口は1000人、1500人、1300人、"大阪"の人口は2000人、2500人とします。}
\NormalTok{df }\OtherTok{\textless{}{-}} \FunctionTok{data.frame}\NormalTok{(}
  \AttributeTok{city =} \FunctionTok{c}\NormalTok{(}\StringTok{"東京"}\NormalTok{, }\StringTok{"大阪"}\NormalTok{, }\StringTok{"東京"}\NormalTok{, }\StringTok{"大阪"}\NormalTok{, }\StringTok{"東京"}\NormalTok{),}
  \AttributeTok{population =} \FunctionTok{c}\NormalTok{(}\DecValTok{1000}\NormalTok{, }\DecValTok{2000}\NormalTok{, }\DecValTok{1500}\NormalTok{, }\DecValTok{2500}\NormalTok{, }\DecValTok{1300}\NormalTok{)}
\NormalTok{)}
\end{Highlighting}
\end{Shaded}

\begin{Shaded}
\begin{Highlighting}[]
\CommentTok{\# 作ったデータを表示してみましょう。}
\FunctionTok{print}\NormalTok{(df)}
\end{Highlighting}
\end{Shaded}

\begin{verbatim}
##   city population
## 1 東京       1000
## 2 大阪       2000
## 3 東京       1500
## 4 大阪       2500
## 5 東京       1300
\end{verbatim}

次に、都市ごとに人口の平均を求めるため、group\_by(city)で都市ごとにデータをまとめ、その後summarise()を使って平均値を計算します。

\begin{Shaded}
\begin{Highlighting}[]
\CommentTok{\# データを都市ごとにまとめてから、その人口の平均を求めて新しいデータフレーム\textquotesingle{}df\_grouped\textquotesingle{}に保存します。}
\NormalTok{df\_grouped }\OtherTok{\textless{}{-}}\NormalTok{ df }\SpecialCharTok{\%\textgreater{}\%}
  \FunctionTok{group\_by}\NormalTok{(city) }\SpecialCharTok{\%\textgreater{}\%}
  \FunctionTok{summarise}\NormalTok{(}\AttributeTok{avg\_population =} \FunctionTok{mean}\NormalTok{(population))}
\end{Highlighting}
\end{Shaded}

\begin{Shaded}
\begin{Highlighting}[]
\CommentTok{\# 平均人口を計算した結果を表示します。}
\FunctionTok{print}\NormalTok{(df\_grouped)}
\end{Highlighting}
\end{Shaded}

\begin{verbatim}
## # A tibble: 2 x 2
##   city  avg_population
##   <chr>          <dbl>
## 1 大阪           2250 
## 2 東京           1267.
\end{verbatim}

以上の操作により、``東京''と''大阪''それぞれの平均人口を簡単に求めることができました。このようにgroup\_by()を使うと、同じカテゴリーに属するデータを一つにまとめて、そのカテゴリーごとの統計値を容易に計算することができます。

\begin{itemize}
\tightlist
\item
  full\_join()は、2つのデータフレームを一つに結合する関数です。この結合方法は全結合(full
  join)と呼ばれ、2つのデータフレームの全ての行を結合します。この時、どちらか一方のデータフレームにしか存在しない行もNAとして保持されます。
\end{itemize}

まずは、2つのサンプルデータフレームを作成します。

\begin{Shaded}
\begin{Highlighting}[]
\CommentTok{\# サンプルデータフレームの作成}
\NormalTok{df1 }\OtherTok{\textless{}{-}} \FunctionTok{data.frame}\NormalTok{(}
  \AttributeTok{city =} \FunctionTok{c}\NormalTok{(}\StringTok{"東京"}\NormalTok{, }\StringTok{"大阪"}\NormalTok{, }\StringTok{"福岡"}\NormalTok{),}
  \AttributeTok{population =} \FunctionTok{c}\NormalTok{(}\DecValTok{1000}\NormalTok{, }\DecValTok{2000}\NormalTok{, }\DecValTok{1500}\NormalTok{)}
\NormalTok{)}

\NormalTok{df2 }\OtherTok{\textless{}{-}} \FunctionTok{data.frame}\NormalTok{(}
  \AttributeTok{city =} \FunctionTok{c}\NormalTok{(}\StringTok{"大阪"}\NormalTok{, }\StringTok{"福岡"}\NormalTok{, }\StringTok{"札幌"}\NormalTok{),}
  \AttributeTok{area =} \FunctionTok{c}\NormalTok{(}\FloatTok{223.00}\NormalTok{, }\FloatTok{340.60}\NormalTok{, }\FloatTok{1121.12}\NormalTok{)}
\NormalTok{)}
\end{Highlighting}
\end{Shaded}

\begin{Shaded}
\begin{Highlighting}[]
\FunctionTok{print}\NormalTok{(df1)}
\end{Highlighting}
\end{Shaded}

\begin{verbatim}
##   city population
## 1 東京       1000
## 2 大阪       2000
## 3 福岡       1500
\end{verbatim}

\begin{Shaded}
\begin{Highlighting}[]
\FunctionTok{print}\NormalTok{(df2)}
\end{Highlighting}
\end{Shaded}

\begin{verbatim}
##   city    area
## 1 大阪  223.00
## 2 福岡  340.60
## 3 札幌 1121.12
\end{verbatim}

次にfull\_join()を使って、これらのデータフレームを結合します。

\begin{Shaded}
\begin{Highlighting}[]
\NormalTok{df\_joined }\OtherTok{\textless{}{-}}\NormalTok{ df1 }\SpecialCharTok{\%\textgreater{}\%}
  \FunctionTok{full\_join}\NormalTok{(df2, }\AttributeTok{by =} \StringTok{"city"}\NormalTok{)}

\FunctionTok{print}\NormalTok{(df\_joined)}
\end{Highlighting}
\end{Shaded}

\begin{verbatim}
##   city population    area
## 1 東京       1000      NA
## 2 大阪       2000  223.00
## 3 福岡       1500  340.60
## 4 札幌         NA 1121.12
\end{verbatim}

この結果、両方のデータフレームの情報を統合することができました。なお、``東京''と''札幌''については片方のデータフレームにしか存在しないため、対応する値がNAとなっています。

\hypertarget{ux30c7ux30fcux30bfux30bbux30c3ux30c8ux306eux5f62}{%
\subsection{データセットの形}\label{ux30c7ux30fcux30bfux30bbux30c3ux30c8ux306eux5f62}}

データは主にワイド形式とロング形式の2つの形式で表されます:

\begin{itemize}
\tightlist
\item
  ワイド形式:各観測値が行、各変数が列として配置される形式。時間経過に伴う複数の測定値などはこの形式で表されます。
\item
  ロング形式:各観測値が行として配置され、変数が列として配置される形式。一つの列には同じ種類の値(例えば全てが体重)が入ります。
\end{itemize}

ワイド形式のデータをロング形式に、またその逆に変換するためには、tidyrパッケージのpivot\_longer()とpivot\_wider()を使用します。

まず、ワイド形式のデータを作成します。

\begin{Shaded}
\begin{Highlighting}[]
\CommentTok{\# ワイド形式のデータフレーム}
\NormalTok{df\_wide }\OtherTok{\textless{}{-}} \FunctionTok{data.frame}\NormalTok{(}
  \AttributeTok{id =} \FunctionTok{c}\NormalTok{(}\DecValTok{1}\NormalTok{, }\DecValTok{2}\NormalTok{),}
  \AttributeTok{weight\_2001 =} \FunctionTok{c}\NormalTok{(}\DecValTok{60}\NormalTok{, }\DecValTok{55}\NormalTok{),}
  \AttributeTok{weight\_2002 =} \FunctionTok{c}\NormalTok{(}\DecValTok{62}\NormalTok{, }\DecValTok{58}\NormalTok{),}
  \AttributeTok{weight\_2003 =} \FunctionTok{c}\NormalTok{(}\DecValTok{63}\NormalTok{, }\DecValTok{59}\NormalTok{)}
\NormalTok{)}

\FunctionTok{print}\NormalTok{(df\_wide)}
\end{Highlighting}
\end{Shaded}

\begin{verbatim}
##   id weight_2001 weight_2002 weight_2003
## 1  1          60          62          63
## 2  2          55          58          59
\end{verbatim}

このデータフレームをロング形式に変換するためには、tidyverseのpivot\_longer()を用います。

\begin{Shaded}
\begin{Highlighting}[]
\CommentTok{\# tidyverseの読み込み}
\FunctionTok{library}\NormalTok{(tidyverse)}
\end{Highlighting}
\end{Shaded}

\begin{verbatim}
## -- Attaching core tidyverse packages ------------------------ tidyverse 2.0.0 --
## v forcats   1.0.0     v stringr   1.5.0
## v lubridate 1.9.2     v tibble    3.2.1
## v purrr     1.0.1     v tidyr     1.3.0
## v readr     2.1.4     
## -- Conflicts ------------------------------------------ tidyverse_conflicts() --
## x dplyr::filter() masks stats::filter()
## x dplyr::lag()    masks stats::lag()
## i Use the conflicted package (<http://conflicted.r-lib.org/>) to force all conflicts to become errors
\end{verbatim}

\begin{Shaded}
\begin{Highlighting}[]
\CommentTok{\# ワイド形式のデータフレーム}
\NormalTok{df\_wide }\OtherTok{\textless{}{-}} \FunctionTok{data.frame}\NormalTok{(}
  \AttributeTok{id =} \FunctionTok{c}\NormalTok{(}\DecValTok{1}\NormalTok{, }\DecValTok{2}\NormalTok{),}
  \AttributeTok{weight\_2001 =} \FunctionTok{c}\NormalTok{(}\DecValTok{60}\NormalTok{, }\DecValTok{55}\NormalTok{),}
  \AttributeTok{weight\_2002 =} \FunctionTok{c}\NormalTok{(}\DecValTok{62}\NormalTok{, }\DecValTok{58}\NormalTok{),}
  \AttributeTok{weight\_2003 =} \FunctionTok{c}\NormalTok{(}\DecValTok{63}\NormalTok{, }\DecValTok{59}\NormalTok{)}
\NormalTok{)}
\end{Highlighting}
\end{Shaded}

\begin{Shaded}
\begin{Highlighting}[]
\CommentTok{\# pivot\_longerを利用してロング形式に変換}
\NormalTok{df\_long }\OtherTok{\textless{}{-}}\NormalTok{ df\_wide }\SpecialCharTok{\%\textgreater{}\%}
  \FunctionTok{pivot\_longer}\NormalTok{(}
    \AttributeTok{cols =} \FunctionTok{starts\_with}\NormalTok{(}\StringTok{"weight"}\NormalTok{),}
    \AttributeTok{names\_to =} \StringTok{"year"}\NormalTok{,}
    \AttributeTok{values\_to =} \StringTok{"weight"}
\NormalTok{  )}
\end{Highlighting}
\end{Shaded}

\begin{Shaded}
\begin{Highlighting}[]
\CommentTok{\# 結果の確認}
\FunctionTok{print}\NormalTok{(df\_long)}
\end{Highlighting}
\end{Shaded}

\begin{verbatim}
## # A tibble: 6 x 3
##      id year        weight
##   <dbl> <chr>        <dbl>
## 1     1 weight_2001     60
## 2     1 weight_2002     62
## 3     1 weight_2003     63
## 4     2 weight_2001     55
## 5     2 weight_2002     58
## 6     2 weight_2003     59
\end{verbatim}

この結果、ワイド形式のデータがロング形式に変換されました。全ての体重の観測値がweight列に、それぞれの観測値がどの年に対応するのかを示す情報がyear列に保存されています。

逆に、ロング形式のデータをワイド形式に変換するには、pivot\_wider()を用います。

\begin{Shaded}
\begin{Highlighting}[]
\CommentTok{\# ロング形式のデータフレームdf\_longを、pivot\_wider()関数を使ってワイド形式に変換します。}
\CommentTok{\# names\_from = yearは、新たに列名として使うデータが格納されている列を指定しています。}
\CommentTok{\# values\_from = weightは、新たに値として使うデータが格納されている列を指定しています。}
\NormalTok{df\_wide\_again }\OtherTok{\textless{}{-}}\NormalTok{ df\_long }\SpecialCharTok{\%\textgreater{}\%}
  \FunctionTok{pivot\_wider}\NormalTok{(}
    \AttributeTok{names\_from =}\NormalTok{ year,}
    \AttributeTok{values\_from =}\NormalTok{ weight}
\NormalTok{  )}
\end{Highlighting}
\end{Shaded}

\begin{Shaded}
\begin{Highlighting}[]
\CommentTok{\# 変換後のワイド形式のデータフレームを確認します。}
\FunctionTok{print}\NormalTok{(df\_wide\_again)}
\end{Highlighting}
\end{Shaded}

\begin{verbatim}
## # A tibble: 2 x 4
##      id weight_2001 weight_2002 weight_2003
##   <dbl>       <dbl>       <dbl>       <dbl>
## 1     1          60          62          63
## 2     2          55          58          59
\end{verbatim}

この結果、ロング形式のデータが元のワイド形式に戻されました。

このように、pivot\_longer()とpivot\_wider()は、データ形状の変換に非常に便利なツールです。これらを使えば、データの視覚化やモデリングのために最適な形状にデータを変換することができます。

以上で、dplyrとtidyrの主な関数の紹介と使用例を終わります。更に詳細な使用方法や他の関数については、以下の参考ページやチートシートをご覧ください:

参考ページ: 1.
\href{http://sugiura-ken.org/wiki/wiki.cgi/exp?page=dplyr}{dplyr wiki}
2. \href{https://stats.biopapyrus.jp/r/tidyverse/dplyr.html}{dplyr
パッケージによるデータ操作と集計} 3.
\href{https://kazutan.github.io/JSSP2018_spring/data_handling.html\#列選択}{R/RStudio入門
列選択} 4.
\href{https://raw.githubusercontent.com/rstudio/cheatsheets/main/translations/japanese/data-wrangling_ja.pdf}{チートシート}

をご参照ください。

\end{document}
