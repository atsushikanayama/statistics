% Options for packages loaded elsewhere
\PassOptionsToPackage{unicode}{hyperref}
\PassOptionsToPackage{hyphens}{url}
%
\documentclass[
]{article}
\usepackage{amsmath,amssymb}
\usepackage{iftex}
\ifPDFTeX
  \usepackage[T1]{fontenc}
  \usepackage[utf8]{inputenc}
  \usepackage{textcomp} % provide euro and other symbols
\else % if luatex or xetex
  \usepackage{unicode-math} % this also loads fontspec
  \defaultfontfeatures{Scale=MatchLowercase}
  \defaultfontfeatures[\rmfamily]{Ligatures=TeX,Scale=1}
\fi
\usepackage{lmodern}
\ifPDFTeX\else
  % xetex/luatex font selection
\fi
% Use upquote if available, for straight quotes in verbatim environments
\IfFileExists{upquote.sty}{\usepackage{upquote}}{}
\IfFileExists{microtype.sty}{% use microtype if available
  \usepackage[]{microtype}
  \UseMicrotypeSet[protrusion]{basicmath} % disable protrusion for tt fonts
}{}
\makeatletter
\@ifundefined{KOMAClassName}{% if non-KOMA class
  \IfFileExists{parskip.sty}{%
    \usepackage{parskip}
  }{% else
    \setlength{\parindent}{0pt}
    \setlength{\parskip}{6pt plus 2pt minus 1pt}}
}{% if KOMA class
  \KOMAoptions{parskip=half}}
\makeatother
\usepackage{xcolor}
\usepackage[margin=1in]{geometry}
\usepackage{color}
\usepackage{fancyvrb}
\newcommand{\VerbBar}{|}
\newcommand{\VERB}{\Verb[commandchars=\\\{\}]}
\DefineVerbatimEnvironment{Highlighting}{Verbatim}{commandchars=\\\{\}}
% Add ',fontsize=\small' for more characters per line
\usepackage{framed}
\definecolor{shadecolor}{RGB}{248,248,248}
\newenvironment{Shaded}{\begin{snugshade}}{\end{snugshade}}
\newcommand{\AlertTok}[1]{\textcolor[rgb]{0.94,0.16,0.16}{#1}}
\newcommand{\AnnotationTok}[1]{\textcolor[rgb]{0.56,0.35,0.01}{\textbf{\textit{#1}}}}
\newcommand{\AttributeTok}[1]{\textcolor[rgb]{0.13,0.29,0.53}{#1}}
\newcommand{\BaseNTok}[1]{\textcolor[rgb]{0.00,0.00,0.81}{#1}}
\newcommand{\BuiltInTok}[1]{#1}
\newcommand{\CharTok}[1]{\textcolor[rgb]{0.31,0.60,0.02}{#1}}
\newcommand{\CommentTok}[1]{\textcolor[rgb]{0.56,0.35,0.01}{\textit{#1}}}
\newcommand{\CommentVarTok}[1]{\textcolor[rgb]{0.56,0.35,0.01}{\textbf{\textit{#1}}}}
\newcommand{\ConstantTok}[1]{\textcolor[rgb]{0.56,0.35,0.01}{#1}}
\newcommand{\ControlFlowTok}[1]{\textcolor[rgb]{0.13,0.29,0.53}{\textbf{#1}}}
\newcommand{\DataTypeTok}[1]{\textcolor[rgb]{0.13,0.29,0.53}{#1}}
\newcommand{\DecValTok}[1]{\textcolor[rgb]{0.00,0.00,0.81}{#1}}
\newcommand{\DocumentationTok}[1]{\textcolor[rgb]{0.56,0.35,0.01}{\textbf{\textit{#1}}}}
\newcommand{\ErrorTok}[1]{\textcolor[rgb]{0.64,0.00,0.00}{\textbf{#1}}}
\newcommand{\ExtensionTok}[1]{#1}
\newcommand{\FloatTok}[1]{\textcolor[rgb]{0.00,0.00,0.81}{#1}}
\newcommand{\FunctionTok}[1]{\textcolor[rgb]{0.13,0.29,0.53}{\textbf{#1}}}
\newcommand{\ImportTok}[1]{#1}
\newcommand{\InformationTok}[1]{\textcolor[rgb]{0.56,0.35,0.01}{\textbf{\textit{#1}}}}
\newcommand{\KeywordTok}[1]{\textcolor[rgb]{0.13,0.29,0.53}{\textbf{#1}}}
\newcommand{\NormalTok}[1]{#1}
\newcommand{\OperatorTok}[1]{\textcolor[rgb]{0.81,0.36,0.00}{\textbf{#1}}}
\newcommand{\OtherTok}[1]{\textcolor[rgb]{0.56,0.35,0.01}{#1}}
\newcommand{\PreprocessorTok}[1]{\textcolor[rgb]{0.56,0.35,0.01}{\textit{#1}}}
\newcommand{\RegionMarkerTok}[1]{#1}
\newcommand{\SpecialCharTok}[1]{\textcolor[rgb]{0.81,0.36,0.00}{\textbf{#1}}}
\newcommand{\SpecialStringTok}[1]{\textcolor[rgb]{0.31,0.60,0.02}{#1}}
\newcommand{\StringTok}[1]{\textcolor[rgb]{0.31,0.60,0.02}{#1}}
\newcommand{\VariableTok}[1]{\textcolor[rgb]{0.00,0.00,0.00}{#1}}
\newcommand{\VerbatimStringTok}[1]{\textcolor[rgb]{0.31,0.60,0.02}{#1}}
\newcommand{\WarningTok}[1]{\textcolor[rgb]{0.56,0.35,0.01}{\textbf{\textit{#1}}}}
\usepackage{graphicx}
\makeatletter
\def\maxwidth{\ifdim\Gin@nat@width>\linewidth\linewidth\else\Gin@nat@width\fi}
\def\maxheight{\ifdim\Gin@nat@height>\textheight\textheight\else\Gin@nat@height\fi}
\makeatother
% Scale images if necessary, so that they will not overflow the page
% margins by default, and it is still possible to overwrite the defaults
% using explicit options in \includegraphics[width, height, ...]{}
\setkeys{Gin}{width=\maxwidth,height=\maxheight,keepaspectratio}
% Set default figure placement to htbp
\makeatletter
\def\fps@figure{htbp}
\makeatother
\setlength{\emergencystretch}{3em} % prevent overfull lines
\providecommand{\tightlist}{%
  \setlength{\itemsep}{0pt}\setlength{\parskip}{0pt}}
\setcounter{secnumdepth}{-\maxdimen} % remove section numbering
\ifLuaTeX
  \usepackage{selnolig}  % disable illegal ligatures
\fi
\IfFileExists{bookmark.sty}{\usepackage{bookmark}}{\usepackage{hyperref}}
\IfFileExists{xurl.sty}{\usepackage{xurl}}{} % add URL line breaks if available
\urlstyle{same}
\hypersetup{
  pdftitle={Rの基礎 by 田辺和奏},
  pdfauthor={まとめ by 金山篤志},
  hidelinks,
  pdfcreator={LaTeX via pandoc}}

\title{Rの基礎 by 田辺和奏}
\author{まとめ by 金山篤志}
\date{2023-06-09}

\begin{document}
\maketitle

{
\setcounter{tocdepth}{2}
\tableofcontents
}
\hypertarget{rstudioux306eux57faux672cux64cdux4f5c}{%
\section{RStudioの基本操作}\label{rstudioux306eux57faux672cux64cdux4f5c}}

RStudioは、R言語のプログラミングをサポートするための統合開発環境(IDE)です。このガイドでは、RStudioの基本的な操作方法について、初心者でも理解しやすいように詳しく解説します。

\hypertarget{rstudioux306eux69cbux6210}{%
\subsection{RStudioの構成}\label{rstudioux306eux69cbux6210}}

まず、RStudioを開くと、主に4つのエリアに分かれた画面が表示されます。それぞれが以下の役割を果たします:

\begin{enumerate}
\def\labelenumi{\arabic{enumi}.}
\tightlist
\item
  スクリプトエディタ(左上):Rのコードを記述し、保存する場所です。ここに書かれたコードは何度でも実行することができ、再利用可能です。
\item
  コンソール(左下):Rのコードを直接実行したり、結果を確認したりする場所です。
\item
  環境/履歴(右上):「環境」タブでは現在のセッションで作成されたオブジェクト(データ、関数など)を確認でき、「履歴」タブでは実行したコードの履歴を見ることができます。
\item
  ファイル/プロット/パッケージ/ヘルプ(右下):このエリアでは、Rのプロジェクトやファイルの管理、グラフの表示、パッケージの管理、Rの関数のヘルプの表示などができます。
\end{enumerate}

\hypertarget{rux30b3ux30deux30f3ux30c9ux306eux5b9fux884c}{%
\subsection{Rコマンドの実行}\label{rux30b3ux30deux30f3ux30c9ux306eux5b9fux884c}}

R言語のstr関数は、データの全体像をつかむための便利なツールです。これを使うと、指定したデータ(変数、データフレーム、リストなど)の全体構造や各部分の詳細を見ることができます。これはまるで、建物の設計図を見るようなもので、その建物がどのように作られているかを理解するのに役立ちます。

例えば、R言語が提供する'iris'というデータセットを使ってみましょう。これはアヤメの花のデータセットで、花弁やがくの長さや幅、そしてアヤメの品種といった情報が含まれています。str関数を使うと、このデータセットがどのように構成されているか、各データがどのような形式で保存されているかを確認することができます。

R言語における「因子」は、このデータセットの中で品種を表す部分です。これはアヤメの品種をラベルとして付けたもので、「setosa」「versicolor」「virginica」の3つの品種が存在します。これらのラベルはそれぞれのアヤメがどの品種に属するかを示しています。そして、データ解析を行う際には、これらのラベル(「因子」)を使って各品種の特徴を比較したりします。たとえば、'setosa'とラベルがつけられたアヤメは他の品種に比べて花びらの長さが短い、などの特性を調べることができます。

具体的には、以下のような情報が表示されます:

\begin{itemize}
\tightlist
\item
  データフレーム(data.frame)
\item
  全部で150の観測値(行)、5つの変数(列)
\item
  各列の名前(Sepal.Length, Sepal.Width, Petal.Length, Petal.Width,
  Species)とそれぞれのデータ型(数値、因子)
\item
  各列の最初の数行のデータ値
\end{itemize}

\begin{Shaded}
\begin{Highlighting}[]
\CommentTok{\# str関数でirisデータセットの構造を確認}
\FunctionTok{str}\NormalTok{(iris)}
\end{Highlighting}
\end{Shaded}

\begin{verbatim}
## 'data.frame':    150 obs. of  5 variables:
##  $ Sepal.Length: num  5.1 4.9 4.7 4.6 5 5.4 4.6 5 4.4 4.9 ...
##  $ Sepal.Width : num  3.5 3 3.2 3.1 3.6 3.9 3.4 3.4 2.9 3.1 ...
##  $ Petal.Length: num  1.4 1.4 1.3 1.5 1.4 1.7 1.4 1.5 1.4 1.5 ...
##  $ Petal.Width : num  0.2 0.2 0.2 0.2 0.2 0.4 0.3 0.2 0.2 0.1 ...
##  $ Species     : Factor w/ 3 levels "setosa","versicolor",..: 1 1 1 1 1 1 1 1 1 1 ...
\end{verbatim}

次に、forループとは何かについて説明します。forループは、指定した範囲の値それぞれに対して、同じ操作を繰り返し行うための構文です。日常生活で例えるならば、「10個のリンゴがあるから、それぞれを洗ってみましょう」というような状況に似ています。

Rでのforループのシンプルな使用例を示します。このスクリプトは1から10までの数値に1を加えて出力します。

\begin{Shaded}
\begin{Highlighting}[]
\CommentTok{\# 「for」ループを始めます。変数「i」に1から10までの数字を順番に入れていきます。}
\CommentTok{\# 要するに、このループは10回繰り返されるわけです。初めてのループでは「i」は1、次に行くと「i」は2になります。}
\CommentTok{\# そのように、「i」の数字は1ずつ増えていきます。}
\ControlFlowTok{for}\NormalTok{ (i }\ControlFlowTok{in} \DecValTok{1}\SpecialCharTok{:}\DecValTok{10}\NormalTok{) \{}

  \CommentTok{\# それぞれのループごとに、数字「i」に1を足した結果を表示します。}
  \CommentTok{\# 例えば、最初のループでは、「i」が1なので、「1 + 1」で結果は2になりますね。}
  \CommentTok{\# そして次に進むと、「i」が2なので、「2 + 1」で結果は3になります。}
  \CommentTok{\# このように、ループが進むごとに「i」に1を足した数字が出力されます。}
  \FunctionTok{print}\NormalTok{(i }\SpecialCharTok{+} \DecValTok{1}\NormalTok{)}
  
\NormalTok{\}}
\end{Highlighting}
\end{Shaded}

\begin{verbatim}
## [1] 2
## [1] 3
## [1] 4
## [1] 5
## [1] 6
## [1] 7
## [1] 8
## [1] 9
## [1] 10
## [1] 11
\end{verbatim}

\begin{Shaded}
\begin{Highlighting}[]
\CommentTok{\# これで、「for」ループを使って1から10までの各数字に1を足して表示する作業は終了です。}
\end{Highlighting}
\end{Shaded}

\hypertarget{ux30aaux30d6ux30b8ux30a7ux30afux30c8ux306eux4f5cux6210ux3068ux78baux8a8d}{%
\subsection{オブジェクトの作成と確認}\label{ux30aaux30d6ux30b8ux30a7ux30afux30c8ux306eux4f5cux6210ux3068ux78baux8a8d}}

Rを使用してオブジェクトを作成し、その内容を確認することができます。以下にいくつかの例を示します。

まず、irisデータセットを''d''という名前のオブジェクトに割り当てます。

\begin{Shaded}
\begin{Highlighting}[]
\CommentTok{\# irisデータセットをオブジェクトdに割り当てる}
\NormalTok{d }\OtherTok{\textless{}{-}}\NormalTok{ iris}
\end{Highlighting}
\end{Shaded}

次に、行列(Matrix)、ベクトル(Vector)、データフレーム(Dataframe)を含むリスト''x''を作成します。

\begin{Shaded}
\begin{Highlighting}[]
\CommentTok{\# 行列、ベクトル、データフレームを含むリストxを作成する}
\NormalTok{x }\OtherTok{\textless{}{-}} \FunctionTok{list}\NormalTok{(}\AttributeTok{m =} \FunctionTok{matrix}\NormalTok{(}\DecValTok{1}\SpecialCharTok{:}\DecValTok{10}\NormalTok{, }\AttributeTok{nrow =} \DecValTok{2}\NormalTok{), }\AttributeTok{v =} \DecValTok{1}\SpecialCharTok{:}\DecValTok{100}\NormalTok{, }\AttributeTok{df =}\NormalTok{ iris)}
\end{Highlighting}
\end{Shaded}

次に、数値かどうかをチェックしてそれらを足し合わせる''tashizan''という名前の関数を定義します。数値でない入力が与えられた場合、関数はエラーを出力します。

\begin{Shaded}
\begin{Highlighting}[]
\CommentTok{\# 関数\textquotesingle{}tashizan\textquotesingle{}を作ります。この関数は、二つの入力(\textquotesingle{}a\textquotesingle{}と\textquotesingle{}b\textquotesingle{})が数値かどうかをチェックし、数値だった場合は足し合わせるというものです。}
\NormalTok{tashizan }\OtherTok{\textless{}{-}} \ControlFlowTok{function}\NormalTok{(a, b) \{}

  \CommentTok{\# まず、\textquotesingle{}a\textquotesingle{}や\textquotesingle{}b\textquotesingle{}が数値かどうかを確認します。}
  \CommentTok{\# \textquotesingle{}!is.numeric(a)\textquotesingle{}や\textquotesingle{}!is.numeric(b)\textquotesingle{}は、「aが数値ではない」あるいは「bが数値ではない」という意味です。}
  \CommentTok{\# \textquotesingle{}||\textquotesingle{}は「または」を意味しますので、この条件は「aが数値ではない、または、bが数値ではない」場合に真となります。}
  \ControlFlowTok{if}\NormalTok{ (}\SpecialCharTok{!}\FunctionTok{is.numeric}\NormalTok{(a) }\SpecialCharTok{||} \SpecialCharTok{!}\FunctionTok{is.numeric}\NormalTok{(b)) \{}

    \CommentTok{\# もし、\textquotesingle{}a\textquotesingle{}や\textquotesingle{}b\textquotesingle{}のいずれかが数値ではない場合、エラーメッセージ「数値を入力してください」を出力して処理を停止します。}
    \CommentTok{\# これは、数値でないものを足し合わせることはできないためです。}
    \FunctionTok{stop}\NormalTok{(}\StringTok{"数値を入力してください"}\NormalTok{)}
\NormalTok{  \}}

  \CommentTok{\# \textquotesingle{}a\textquotesingle{}も\textquotesingle{}b\textquotesingle{}も数値である場合は、\textquotesingle{}a\textquotesingle{}と\textquotesingle{}b\textquotesingle{}を足し合わせた結果を出力します。}
\NormalTok{  a }\SpecialCharTok{+}\NormalTok{ b}
\NormalTok{\}}
\end{Highlighting}
\end{Shaded}

RStudioのコード補完機能とは何かというと、これはコーディング中にあなたが入力しようとしているコードを予測し、補完してくれる機能のことを言います。たとえば、変数名を打ち始めたときに、その変数名の残りの部分をRStudioが自動的に予測して表示してくれます。

具体的な例を挙げてみましょう。以下に示した2つのコードについて説明します。

1つ目のコードは:

\begin{Shaded}
\begin{Highlighting}[]
\CommentTok{\# forループを用いて1から10までの数値を出力}
\ControlFlowTok{for}\NormalTok{ (i }\ControlFlowTok{in} \DecValTok{1}\SpecialCharTok{:}\DecValTok{10}\NormalTok{) \{}
  \FunctionTok{print}\NormalTok{(i)}
\NormalTok{\}}
\end{Highlighting}
\end{Shaded}

\begin{verbatim}
## [1] 1
## [1] 2
## [1] 3
## [1] 4
## [1] 5
## [1] 6
## [1] 7
## [1] 8
## [1] 9
## [1] 10
\end{verbatim}

このコードでは、forという単語を入力し始めると、RStudioはforループを使いたいのかもしれないと予測します。そのため、あなたがforと打つと、その後にループの中で使われる変数(ここではi)とループの範囲(ここでは1:10)を入力する場所を示す一部のコードを補完してくれます。

2つ目のコードは:

\begin{Shaded}
\begin{Highlighting}[]
\CommentTok{\# 変数xを定義し、その結果をdata.frameオブジェクトに格納}
\NormalTok{x }\OtherTok{\textless{}{-}} \DecValTok{1} \SpecialCharTok{+} \DecValTok{1}
\NormalTok{jobs\_demo\_results }\OtherTok{\textless{}{-}} \FunctionTok{data.frame}\NormalTok{(}\AttributeTok{x =}\NormalTok{ x)}
\FunctionTok{print}\NormalTok{(jobs\_demo\_results}\SpecialCharTok{$}\NormalTok{x)}
\end{Highlighting}
\end{Shaded}

\begin{verbatim}
## [1] 2
\end{verbatim}

ここでは、data.frameやprintといった関数名を打ち始めると、RStudioはあなたがこれらの関数を使いたいと予測します。そのため、関数名を打ち始めると、それぞれの関数が必要とする引数(ここではx
=
xやjobs\_demo\_results\$x)を入力する場所を示す一部のコードを補完してくれます。

このように、RStudioのコード補完機能は、入力を予測してくれるため、コーディングがよりスムーズになり、またタイプミスを減らすことができます。また、関数の名前や引数が思い出せない時でも、部分的な入力から予測してくれるため、記憶を助ける役割も果たしてくれます。

\hypertarget{rux3067ux306eux30d5ux30a1ux30a4ux30ebux306eux8aadux307fux66f8ux304d}{%
\section{Rでのファイルの読み書き}\label{rux3067ux306eux30d5ux30a1ux30a4ux30ebux306eux8aadux307fux66f8ux304d}}

このセクションでは、Rを使用してさまざまな種類のファイルを読み書きするプロセスについて詳しく説明します。これにはCSVファイル、Excelファイル、SAS、SPSS、STATAなどのデータファイルが含まれます。

\hypertarget{csvux30d5ux30a1ux30a4ux30ebux306eux8aadux307fux66f8ux304d}{%
\subsection{CSVファイルの読み書き}\label{csvux30d5ux30a1ux30a4ux30ebux306eux8aadux307fux66f8ux304d}}

CSV(Comma-Separated
Values)ファイルは、Rでよく使用されるデータ形式です。このコンマ区切りのテキストファイル形式では、各行がデータセットの1つの観測値を表し、各列が変数を示します。

irisデータをCSVファイルに書き込み、その後読み込む方法を見てみます。

\begin{Shaded}
\begin{Highlighting}[]
\CommentTok{\# irisデータをCSVファイルに書き込む}
\FunctionTok{write.csv}\NormalTok{(iris, }\AttributeTok{file =} \StringTok{"iris.csv"}\NormalTok{)}

\CommentTok{\# 書き込んだCSVファイルを読み込む}
\NormalTok{data }\OtherTok{\textless{}{-}} \FunctionTok{read.csv}\NormalTok{(}\StringTok{"iris.csv"}\NormalTok{)}
\end{Highlighting}
\end{Shaded}

\begin{Shaded}
\begin{Highlighting}[]
\CommentTok{\# \textquotesingle{}head\textquotesingle{}関数を使って、\textquotesingle{}data\textquotesingle{}データセットの最初の6行を表示します。}
\FunctionTok{head}\NormalTok{(data)}
\end{Highlighting}
\end{Shaded}

\begin{verbatim}
##   X Sepal.Length Sepal.Width Petal.Length Petal.Width Species
## 1 1          5.1         3.5          1.4         0.2  setosa
## 2 2          4.9         3.0          1.4         0.2  setosa
## 3 3          4.7         3.2          1.3         0.2  setosa
## 4 4          4.6         3.1          1.5         0.2  setosa
## 5 5          5.0         3.6          1.4         0.2  setosa
## 6 6          5.4         3.9          1.7         0.4  setosa
\end{verbatim}

\hypertarget{excelux30d5ux30a1ux30a4ux30ebux306eux8aadux307fux66f8ux304d}{%
\subsection{Excelファイルの読み書き}\label{excelux30d5ux30a1ux30a4ux30ebux306eux8aadux307fux66f8ux304d}}

Excelファイルは、ビジネスや研究においても広く使用されており、Rでは便利なツールが提供されています。

readxlパッケージを使用してExcelファイルを読み込む方法を示します。このために、パッケージに含まれるサンプルデータを使用します。

\begin{Shaded}
\begin{Highlighting}[]
\CommentTok{\# readxlパッケージを読み込む}
\FunctionTok{library}\NormalTok{(readxl)}

\CommentTok{\# Excelファイルのパスを取得する}
\NormalTok{path }\OtherTok{\textless{}{-}} \FunctionTok{system.file}\NormalTok{(}\StringTok{"extdata"}\NormalTok{, }\StringTok{"datasets.xlsx"}\NormalTok{, }\AttributeTok{package =} \StringTok{"readxl"}\NormalTok{)}

\CommentTok{\# Excelファイルを読み込む}
\NormalTok{data }\OtherTok{\textless{}{-}} \FunctionTok{read\_excel}\NormalTok{(path)}
\end{Highlighting}
\end{Shaded}

\begin{Shaded}
\begin{Highlighting}[]
\CommentTok{\# \textquotesingle{}head\textquotesingle{}関数を使って、\textquotesingle{}data\textquotesingle{}データセットの最初の6行を表示します。}
\FunctionTok{head}\NormalTok{(data)}
\end{Highlighting}
\end{Shaded}

\begin{verbatim}
## # A tibble: 6 x 5
##   Sepal.Length Sepal.Width Petal.Length Petal.Width Species
##          <dbl>       <dbl>        <dbl>       <dbl> <chr>  
## 1          5.1         3.5          1.4         0.2 setosa 
## 2          4.9         3            1.4         0.2 setosa 
## 3          4.7         3.2          1.3         0.2 setosa 
## 4          4.6         3.1          1.5         0.2 setosa 
## 5          5           3.6          1.4         0.2 setosa 
## 6          5.4         3.9          1.7         0.4 setosa
\end{verbatim}

\hypertarget{sasspssstataux30d5ux30a1ux30a4ux30ebux306eux8aadux307fux66f8ux304d}{%
\subsection{SAS、SPSS、STATAファイルの読み書き}\label{sasspssstataux30d5ux30a1ux30a4ux30ebux306eux8aadux307fux66f8ux304d}}

SAS、SPSS、STATAファイルは、社会科学や統計分析の分野でよく使用されます。今回はRのhavenパッケージを使用して、インターネットから直接これらのファイルを読み込む方法を説明します。以下にSASファイルを例に示します。

まず、havenパッケージをインストールして読み込む必要があります。havenパッケージは、SAS、SPSS、Stataといった統計ソフトウェアのデータファイルをRで読み込むためのパッケージです。

\begin{Shaded}
\begin{Highlighting}[]
\CommentTok{\# havenパッケージを読み込む}
\FunctionTok{library}\NormalTok{(haven)}

\CommentTok{\# SASファイルのパスを取得する}
\NormalTok{path }\OtherTok{\textless{}{-}} \FunctionTok{system.file}\NormalTok{(}\StringTok{"examples"}\NormalTok{, }\StringTok{"iris.sas7bdat"}\NormalTok{, }\AttributeTok{package =} \StringTok{"haven"}\NormalTok{)}

\CommentTok{\# SASファイルを読み込む}
\NormalTok{data }\OtherTok{\textless{}{-}} \FunctionTok{read\_sas}\NormalTok{(path)}
\end{Highlighting}
\end{Shaded}

\begin{Shaded}
\begin{Highlighting}[]
\CommentTok{\# \textquotesingle{}head\textquotesingle{}関数を使って、\textquotesingle{}data\textquotesingle{}データセットの最初の6行を表示します。}
\FunctionTok{head}\NormalTok{(data)}
\end{Highlighting}
\end{Shaded}

\begin{verbatim}
## # A tibble: 6 x 5
##   Sepal_Length Sepal_Width Petal_Length Petal_Width Species
##          <dbl>       <dbl>        <dbl>       <dbl> <chr>  
## 1          5.1         3.5          1.4         0.2 setosa 
## 2          4.9         3            1.4         0.2 setosa 
## 3          4.7         3.2          1.3         0.2 setosa 
## 4          4.6         3.1          1.5         0.2 setosa 
## 5          5           3.6          1.4         0.2 setosa 
## 6          5.4         3.9          1.7         0.4 setosa
\end{verbatim}

\hypertarget{rux3084rstudioux3067ux56f0ux3063ux305fux3068ux304dux306eux5bfeux51e6ux6cd5}{%
\section{RやRStudioで困ったときの対処法}\label{rux3084rstudioux3067ux56f0ux3063ux305fux3068ux304dux306eux5bfeux51e6ux6cd5}}

Rの関数の詳細を知りたいとき、または使い方について学びたいときには、「?」または「help」関数を利用します。これらの関数は、Rの関数の説明や使い方を表示するためのツールです。

\begin{Shaded}
\begin{Highlighting}[]
\NormalTok{?plot  }\CommentTok{\# \textquotesingle{}plot\textquotesingle{}関数の説明と使い方を表示します。}
\end{Highlighting}
\end{Shaded}

\begin{verbatim}
## Help on topic 'plot' was found in the following packages:
## 
##   Package               Library
##   base                  /Library/Frameworks/R.framework/Resources/library
##   graphics              /Library/Frameworks/R.framework/Versions/4.3-arm64/Resources/library
## 
## 
## Using the first match ...
\end{verbatim}

\begin{Shaded}
\begin{Highlighting}[]
\FunctionTok{help}\NormalTok{(plot)  }\CommentTok{\# 同様に、\textquotesingle{}help\textquotesingle{}関数も\textquotesingle{}plot\textquotesingle{}関数の説明と使い方を表示します。}
\end{Highlighting}
\end{Shaded}

\begin{verbatim}
## Help on topic 'plot' was found in the following packages:
## 
##   Package               Library
##   base                  /Library/Frameworks/R.framework/Resources/library
##   graphics              /Library/Frameworks/R.framework/Versions/4.3-arm64/Resources/library
## 
## 
## Using the first match ...
\end{verbatim}

Rのパッケージには、そのパッケージの使用方法を詳しく説明したチュートリアルが含まれていることがあります。これらのチュートリアルは「ビネット」(vignette)と呼ばれ、vignette関数を使って表示することができます。

\begin{Shaded}
\begin{Highlighting}[]
\FunctionTok{vignette}\NormalTok{(}\StringTok{"dplyr"}\NormalTok{)  }\CommentTok{\# \textquotesingle{}dplyr\textquotesingle{}パッケージのビネット(チュートリアル)を開きます。}
\end{Highlighting}
\end{Shaded}

\begin{verbatim}
## starting httpd help server ... done
\end{verbatim}

上記のコードは、``dplyr''パッケージのビネットを開く方法を示しています。これにより、``dplyr''パッケージの詳しい使用方法を理解することができます。

\end{document}
